\documentclass[]{article}
\usepackage{lmodern}
\usepackage{amssymb,amsmath}
\usepackage{ifxetex,ifluatex}
\usepackage{fixltx2e} % provides \textsubscript
\ifnum 0\ifxetex 1\fi\ifluatex 1\fi=0 % if pdftex
  \usepackage[T1]{fontenc}
  \usepackage[utf8]{inputenc}
\else % if luatex or xelatex
  \ifxetex
    \usepackage{mathspec}
  \else
    \usepackage{fontspec}
  \fi
  \defaultfontfeatures{Ligatures=TeX,Scale=MatchLowercase}
\fi
% use upquote if available, for straight quotes in verbatim environments
\IfFileExists{upquote.sty}{\usepackage{upquote}}{}
% use microtype if available
\IfFileExists{microtype.sty}{%
\usepackage{microtype}
\UseMicrotypeSet[protrusion]{basicmath} % disable protrusion for tt fonts
}{}
\usepackage[margin=1in]{geometry}
\usepackage{hyperref}
\hypersetup{unicode=true,
            pdftitle={Canada's Labour Force - Analysis},
            pdfauthor={Sherry Luo and Tony Wang},
            pdfborder={0 0 0},
            breaklinks=true}
\urlstyle{same}  % don't use monospace font for urls
\usepackage{graphicx,grffile}
\makeatletter
\def\maxwidth{\ifdim\Gin@nat@width>\linewidth\linewidth\else\Gin@nat@width\fi}
\def\maxheight{\ifdim\Gin@nat@height>\textheight\textheight\else\Gin@nat@height\fi}
\makeatother
% Scale images if necessary, so that they will not overflow the page
% margins by default, and it is still possible to overwrite the defaults
% using explicit options in \includegraphics[width, height, ...]{}
\setkeys{Gin}{width=\maxwidth,height=\maxheight,keepaspectratio}
\IfFileExists{parskip.sty}{%
\usepackage{parskip}
}{% else
\setlength{\parindent}{0pt}
\setlength{\parskip}{6pt plus 2pt minus 1pt}
}
\setlength{\emergencystretch}{3em}  % prevent overfull lines
\providecommand{\tightlist}{%
  \setlength{\itemsep}{0pt}\setlength{\parskip}{0pt}}
\setcounter{secnumdepth}{0}
% Redefines (sub)paragraphs to behave more like sections
\ifx\paragraph\undefined\else
\let\oldparagraph\paragraph
\renewcommand{\paragraph}[1]{\oldparagraph{#1}\mbox{}}
\fi
\ifx\subparagraph\undefined\else
\let\oldsubparagraph\subparagraph
\renewcommand{\subparagraph}[1]{\oldsubparagraph{#1}\mbox{}}
\fi

%%% Use protect on footnotes to avoid problems with footnotes in titles
\let\rmarkdownfootnote\footnote%
\def\footnote{\protect\rmarkdownfootnote}

%%% Change title format to be more compact
\usepackage{titling}

% Create subtitle command for use in maketitle
\providecommand{\subtitle}[1]{
  \posttitle{
    \begin{center}\large#1\end{center}
    }
}

\setlength{\droptitle}{-2em}

  \title{Canada's Labour Force - Analysis}
    \pretitle{\vspace{\droptitle}\centering\huge}
  \posttitle{\par}
    \author{Sherry Luo and Tony Wang}
    \preauthor{\centering\large\emph}
  \postauthor{\par}
      \predate{\centering\large\emph}
  \postdate{\par}
    \date{16/12/2019}


\begin{document}
\maketitle

\hypertarget{part-i-job-tenure-in-canada}{%
\section{Part I: Job Tenure in
Canada}\label{part-i-job-tenure-in-canada}}

For this project, we conduct an exploratory analysis on job tenure data
collected from Statistics Canada from 1976-2018. Though there are many
ways to look at employment data, with job security and the so-called
``gig economy'' as a topic of discussion in recent years
\cite{NBERw22667}, we wish to start by seeing how the length of
employment of full-time workers has changed over time in Canada by
gender and age. Job tenure measures the length of time an employee has
been employed by their employer. For this analysis, job tenure was
categorized into three groups: short-term (less than 1 year of
employment), intermediate (1 to 5 years) and long-term (more than 5
years).

\hypertarget{back-to-back-pyramid-plot}{%
\subsection{Back-to-back Pyramid Plot}\label{back-to-back-pyramid-plot}}

\hypertarget{story}{%
\subsubsection{Story:}\label{story}}

We start the analysis with an animated pyramid plot to compare changes
in job tenure by gender from 1976-2018. It appears that job tenure has
increased for both genders, and a higher number of men appear in each of
three groups compared to women. Additionally, changes in job tenure also
depends on financial state of Canada. Generally, when the economy is
stable, the number of short-term jobs increase and transitional stages
(moving from short-term to intermediate and intermediate to long-term
positions) follow accordingly. However, when a financial crisis strikes
such as the recession in 2008-2012, the number of short-term jobs
decrease, but the transition stages in both the intermediate and
long-term group remain relatively unchanged (i.e.~layoffs affect workers
with lower seniorty more compared to workers with higher seniority). The
pyramid plot is a nice visual but comparing the distributions of men and
women is not obvious and difficult to do. But due to the animated
component, each year can only be viewed individually not collectively
(hence the possibility of being wrong, as we will see).

\hypertarget{method}{%
\subsubsection{Method:}\label{method}}

The cansim package allowed us to easily obtain the data from Statistics
Canada. For cleaning, we filtered the necessary data and relabel/order
the age and job tenure brackets for clarity. To create a symmetric
back-to-back plot, the female values were multiplied by negative one. In
terms of design, we selected high contrast colors to compare males and
females and added direct labelling. However, as mentioned previously,
comparing the distribution over time was not easy. Thus it may have been
better to overlay the distributions rather than back-to-back as
suggested by Dr.Bolker. The resulting plot will be less attractive but
will be more informative for comparison. For more control on playback,
we saved the plot as a MP4 and if played continuously each frame is
played for 2 seconds (fps=2). Finally, for visual purposes we assumed
the bin width to be equal for each job tenure bracket.

\hypertarget{average-tenure-plots}{%
\subsection{Average Tenure Plots}\label{average-tenure-plots}}

\hypertarget{story-1}{%
\subsubsection{Story:}\label{story-1}}

We created two plots based on the average job tenure. In the first plot,
we examined the average tenure by gender and more detailed trends are
presented. We see a huge gap between men and women but with time this
gap narrowed around the early 2000s due to more women entering the work
force around the early 1990s. There are more men in the workforce as
indicated by similarity in trend for ?males? and ?both sex?. Comparing
men and women, the average tenure for women increased steadily over time
but decreased for men after 1996. However, this plot may be misleading
as it does not consider the various age groups of each gender. To
address this problem, it is best to separate the age groups for each
gender. In this second plot, the average job tenure by gender and age
are examined. For women, we see that for all age groups except 15 to 24
year old age brackets, there is a sight increase in the average job
tenure over time. For men aged 65 years or more, job tenure increases
until 1996 and substantially decreases afterwards. We also see a similar
trend in 55 to 64 year old males, though the decrease is very marginal
compared to 65 years and older. This slightly contradicts our
observation from before since job tenure actually decreased for men
after 1996.

\hypertarget{method-1}{%
\subsubsection{Method:}\label{method-1}}

The two interactive plots were created with plotly. For both plots, we
opted for high contrast colors and used the same color palette for
aesthetic consistency while trying to keep the plots accessible to the
visually impaired. In the first plot, average tenure for men was
represented in blue and dark pink for females. We felt purple would be
more visually obvious to represent ``halfway between'' since average
tenure for both sexes lie between males and females. In the second plot,
the coloring scheme highlights the ordering of the age categories.
Facets were used to allow a side-by-side comparison of trends for each
gender and age group. We used a linear axis as the discrepancies between
groups were substantial and a log-scale would have been rather deceptive
when comparing the differences between groups. Additionally, units were
rounded to tenth of the year as years is more comprehensible to the
general audience. The interactive elements allowed the data to be
explored in better detail without adding excessive cluster to the plots.

(Tony,I'm trying to add line width to represent cohort size for each age
group, I'll let you know if I figure this out)

\hypertarget{sunburst-plot}{%
\subsection{Sunburst Plot}\label{sunburst-plot}}

\hypertarget{story-2}{%
\subsubsection{Story:}\label{story-2}}

Finally, we conclude the analysis with a sunburst plot. Initially,
cumulatively from 1976-2018, the workforce is 40\% female and 60\% male.
As we move upwards to each ring level, we can isolate and analyze
specific combinations of the data. With this ordering, we can explore
the age composition for each sex at different job tenure levels for
different years. For example, among female workers who have been at
their job for 1 to 5 years in 2009, majority are between 25 and 54 years
old. This scenario can also be examined for men and we see that in 2009,
male workers who have been at their job for 1-5 years are mostly between
25 and 54 years old, mirroring the trend that we saw in female workers.

\hypertarget{method-2}{%
\subsubsection{Method:}\label{method-2}}

Sunburst plot was created with the sunburstR package. As Dr.Bolker has
stated, these plots are heavily dependent on the ordering of the
variables. To suit our analysis above, we selected the ordering to be
gender, job tenure, specific year then age group. We added a tooltip
that allows a user to select a specific wedge of the plot and the
breakdown can be seen on the right. We did consider rounding the numeric
values in the tooltip, however, customizing it required knowledge of
javascript. The d3.js categorical color scheme was used here.

\hypertarget{part-ii-gender-pay-gap-analysis}{%
\section{Part II: Gender Pay Gap
Analysis}\label{part-ii-gender-pay-gap-analysis}}

For the second part of the analysis, we will look at some specific
aspects of the workforce in recent years. One topic that frequently
comes up when people discuss labour data is the gender pay gap, which
can generally be described as a discrepancy between how much men and
women get paid for comparable work. Everyone has heard about the 70
cents to the dollar saying, but considering how people's pay differ
depending on fields of study, education levels, region, and many other
features, it is beneficial to take a precursory glance at these sorts of
variations to see how things actually work.

\hypertarget{geospatial-facet-plot}{%
\subsection{Geospatial Facet Plot}\label{geospatial-facet-plot}}

\hypertarget{story-3}{%
\subsubsection{Story:}\label{story-3}}

We begin by looking at the gender pay gap through the creation of an
interactive, multi-faceted geospatial plot, coloured by the predominant
field for each workfoace cohort. From this plot, we see that:

\begin{itemize}
\tightlist
\item
  At the College-level and below, business is the predominant field of
  study for all provinces and Yukon, and the trades/natural resources is
  the most common field in the NorthWest Territories and Nunavut.
\item
  At the University Certificate level, the most populus provinces in
  Canada (Ontario, Quebec, British Columbia and Alberta) have business
  as the most common field of study, with the rest of the provinces and
  territories having Education as the most common field.
\item
  At the Bachelor's degree level, Social sciences become the most common
  field in Ontario, British Columbia and the Yukon, and Business becomes
  the most common field of study in New Brunswick, but for all other
  provinces, the field of study remains the same as the University
  Certificate level.
\item
  Geographical Industry trends at the Master's degree level largely
  resemble those at the University Certificate level, with the exception
  of New Brunswick and Manitoba
\item
  At the Doctorate level, sciences are the single most common field of
  study in every province and territory; however, it is interesting to
  note that this is for the physical sciences (i.e.~biology, chemistry,
  and physics), meaning that this does not include mathematics and
  computer sciences.
\end{itemize}

Now that we have an idea of what the most common fields of study are in
different provinces and territories, we can look at the gender balance
within each of these fields as well as any gender pay gaps. They can be
briefly summarized as follows:

\begin{itemize}
\tightlist
\item
  Where Business and administration is the predominant field, females
  generally make up between 40 to 80 percent of that particular cohort,
  with the proportion of females generally declining as the education
  level goes up. There appears to be a universal gender pay gap across
  all provinces that goes anywhere between 7\% (minimum value, in Prince
  Edward Island at the Bachelor's level) to around 24\% (as observed in
  Alberta at the Master's degree level).
\item
  In provinces and territories where Education is the most common field
  of study, females make up at least 60\% of that particular cohort. For
  people with Master's degrees, Nunavut and Prince Edward Island have
  slightly greater median pay levels for females than males, though the
  difference is very small. In other areas where education is the most
  common field, the median salary of males is greater than that of
  females, though this difference ranges from slight (Nova Scotia,
  Master's degree level) to substantial (e.g.~Saskatchewan, University
  certificate level).
\item
  Science disciplines are the predominant field of study at the
  doctorate level across all genders. In all provinces, females make up
  between 30 to 45 percent of the Science PhD cohort, and with the
  exception of Nova Scotia, males have a greater median salary in all
  provinces than females. The territories have so few Sciences PhDs that
  with the exception of males in the Northwest Territories, Statistics
  Canada actually does not provide wage data. No inferences can be made
  for these regions given the small number of relevant individuals.
\item
  Social sciences are predominant at the Bachelor's degree level for
  only two provinces and one territory. These cohorts are roughly 60\%
  female and there appears to be a consistent and rather sizable gender
  pay gap in favour of males across all three of these jurisdictions.
\item
  The trades, services, natural resources and conservation fields have
  very low female representation (roughly 30\%), and a pay gap of either
  10 percent (Nunavut) or 25 percent (Northwest Territories) in favor of
  males.
\end{itemize}

\hypertarget{method-3}{%
\subsubsection{Method:}\label{method-3}}

To look at gender pay gaps, we combined industry and wage data obtained
from the 2016 Census into an interactive, multi-faceted geospatial plot,
coloured by the predominant field for each workfoace cohort. The data we
used comes from two datasets of the 2016 Census, one of which covered
wages of individuals by gender and field of study (Catalogue
No.~98-400-X2016280), and one of which tallied the number of employees
by various factors such as industry (Catalogue No.~98-402-X2016010-T4),
gender, field of study and earned qualifications. The wage data from
statistics Canada was extracted from the source data, a 6GB CSV file,
using a workstation desktop with the necessary system resources to do
so, and all plots were used on the reduced dataset to reduce system
resource usage. These datasets were combined and used to create a five
facet geospatial plot using the simplified digital boundary files. The
use of digital boundary files rather than detailed cartographic files
accelerates the rendering process and reduces the final file size,
albeit at the cost of some information about the northern territories'
islands. For the purposes of this analysis, this appears to be fine.

The viridis ``C'' or ``plasma'' color scale was used for brightness and
high contrast, not uniformity. Though the levels of the factors are not
directional, these colors are still bright enough and distinctive enough
such that they are suitable for distinguishing the different fields. It
was decided to facet the geospatial data by education levels in order to
best see how education levels may be related to the fields and location,
and the rest of the information on cohort sizes, female representation,
and salaries by field and gender were added into the hover tooltip
through the use of several dummy aesthetics.

\hypertarget{advantages-and-disadvantages}{%
\subsubsection{Advantages and
Disadvantages:}\label{advantages-and-disadvantages}}

The advantage of the spatial plot is that it allows people to
immediately identify the largest educational cohort within each
province, and see some useful information about demographics and median
pay within that cohort should they desire to. That is, this graphic
provides a very brief high-level overview for those who are not
interested in aggregated national level data, but want to get an idea of
what they may expect when they look at regional-level data. However, we
do acknowledge that doing things this way, while being a non-invasive
way to present large amounts of data on the most common industries
within each province or territory, does have limitations:

\begin{itemize}
\tightlist
\item
  First, this plot only shows relevant information for the largest field
  of study. The issue here is that if there are two educational cohorts
  that both make up, for example, around 40 percent of the total
  population for that province, the smaller of the two will not be
  shown.
\item
  Secondly, there is currently no built-in mechanism to simultaneously
  pinpoint multiple entries at once. This makes it tedious to make
  multiple comparisons.
\item
  Third, there is an argument to be made that province-level data may
  not be as relevant as the national-level difference between different
  age groups.
\item
  As a technical point, the hover only really works when you go over the
  edges of the polygon. We suspect this is an issue with how ggplotly
  interacts with ggplot in the creation of traces, though we were
  unfortunately unable to correct this behaviour.
\end{itemize}

Overall, it seems that we do need to take a more in-depth look at the
situation given what we have seen.

\hypertarget{facet-dot-plot}{%
\subsection{Facet Dot Plot}\label{facet-dot-plot}}

\hypertarget{story-4}{%
\subsubsection{Story}\label{story-4}}

Given the considerations we noted above, it was decided that a single
static plot, with multiple facets for different industries, age groups,
and education levels for national-level data, would allow us to somewhat
address all of the issues at hand from the initial exploratory plot on
regional-level data. Data was sourced again frome the 2016 census,
except this time, age information was retained and since we are no
longer interested in cohort sizes, it was possible to source all of the
information from the 2016 Census, Catalogue No.~98-400-X2016280
(Statistics Canada, 2018). From the facet plot, we see that:

\begin{itemize}
\tightlist
\item
  The gender pay gap seems to be the least prominent for the 15 to 29
  year old age cohort. Whele the differences are slightly in favour of
  Males at the College level, and somewhat in favour of females at the
  Doctorate level, the difference at other levels of Education for this
  cohort, while technically in favour of males, is extremely small.
\item
  For those aged 30 to 59, there is a consistent pay gap in favour of
  males across virtually every industry and education level up to the
  Master's degree level. While there are some exceptions where females
  are paid similarly, the gender pay gap appears fairly persistent
  across a variaety of circumstances. While there is a pay gap at the
  doctorate level that generally favours males, the picture is not as
  clear.
\item
  At the 60+ age group, we notice that the gender pay gaps are still
  present, but are far more erratic than those in other age groups.
  While pay gaps at different educational levels still seem to favour
  males, these trends are much less consistent across both educational
  lebels and industries.
\item
  In terms of industries, health care at the University Certificate,
  Bachelor's degree and Master's degree consistently offers comparable
  pay to both genders. However, at the College level, there is a
  persistent pay gap in favour of males at all education levels, while
  at the doctorate level and adove the 15-29 year age brackets, there is
  a persistent pay gap in favour of females that enlarges at the 60+ age
  group. Otherwise, there are few consistent trends with respect to
  industry.
\end{itemize}

Overall, based on the national level data, we see that there is a
persistent gender pay gap across a variety of settings and industries,
but the picture is not very clear cut. It seems that the strongest trend
appears to be the appearence of a pay gap for those at the university
level after the 15 to 29 year bracket that persists well into the
working years, but narrows as workers approach the age of 60. It is
entirely possible that the observed trends are the result of some sort
of ``motherhood penalty'', where younger women upon having children get
treated differently and ``lose out'' on pay increases and promotion
\cite{budig2001wage}, but work experience gained over time may reduce
these differences as workers get older. A more detailed analysis will be
needed if we are to check the validity of this conjecture.

\hypertarget{method-4}{%
\subsubsection{Method}\label{method-4}}

For the purposes of this analysis, we are simply interested in whether
there is a consistent pay gap or not based on the factors that are most
likely to cause pay variations. Given that there are a large range of
possible salaries for different fields of study and educational levels,
a log scale was chosen for the horizontal axes for income. Faceting was
done by placing education levels as parallel rows and age groups
vertically, as each facet by itself is read left to right and sailent
comparisons are generally not made between education levels for this
topic, but rather between different age groups within the same education
level. Colors were selected on the basis of common associations for each
gender and to maximize contrast on the background, which was darkened
and spaced out to help the reader distinguish between separate facets.
Gridlines were included by default and kept as they seem to help
somewhat in tracking which values belong to which fields. Finally,
fields were left in reverse alphabetical order as there is no way to
consistently and naturally order this particular factor. This plot has
the advantage of simplicity and clarity at the expense of losing
regional-level information in favour of national-level information, but
as region may not be as important of a predictor in the preliminary
stages of exploratory analysis, this trade-off may be worthwhile for the
detailed information it provides.

\bibliography{citations}
\bibliographystyle{abbrv}


\end{document}
